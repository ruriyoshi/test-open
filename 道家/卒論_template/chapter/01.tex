\chapter{序論}

\section{核融合発電}
化石エネルギーからの脱却が叫ばれる現代社会において、火力発電所の二酸化炭素排出量はもちろんのこと、原子力発電所の放射性廃棄物と有事の際の安全性に関する懸念、太陽光発電や風力発電の効率立地依存性といった現在の主要発電手法の抱える様々な問題を克服する新たなエネルギー源の開発は地球規模で注目されるべき課題である。その中で、核融合発電は高い環境適応性と持続可能性を持ちながら十分な発電量を確保できるエネルギー問題の最終解たりうる技術であって、その原理は質量数の小さい原子核同士が衝突し結合する際の質量欠損を生成物の熱として取り出すものである。核融合の反応条件は反応物の組み合わせによって様々ではあるが最も反応断面積あたりの温度が小さいD-T反応が現在世界的な目標とされており、重水素(Deuterium)は海水から、三重水素(Tritium)は炉中のリチウムブランケットと高速中性子との反応から生成することで実現される。それぞれの反応式は以下の通り表され\cite{plasma}、リチウムを海水から取り出すことができれば海水という不偏、かつほぼ無限の資源からすべての反応を賄うことが可能となり、これが核融合発電が注目される大きな理由の一つである。
\begin{align}
\text{D + T}&\longrightarrow{}^{4}\text{He (3.52 MeV) + n (14.06 MeV)}\\
{}^{6}\text{Li + n}&\longrightarrow{}^{4}\text{He + T + 4.8 MeV}\\
{}^{7}\text{Li + n}&\longrightarrow{}^{4}\text{He + T + n + 2.5 MeV}
\end{align}
\begin{figure}[H]\centering\includegraphics[width=0.6\columnwidth]{/fig/system.png}
	\caption{核融合発電システムの概要\cite{PF}}\label{fig:system}
\end{figure}

これら燃料となる粒子は正電荷をもつため反応にはクーロン反発力に抗するだけの大きな運動エネルギーが与えられる環境が必要となるが、その方法の一つとして有力なのは磁場閉じ込め方式である。これは電離プラズマを磁力線の容器に一定時間高い圧力のもと保持することで粒子間の衝突頻度を高め核融合を期待するものであって、その磁力線構造はミラー型、ヘリカル型、逆転磁場配位FRC等様々ではあるが、国際熱核融合実験炉ITERにも採用されているトカマク型が最も実現性の高いものとされている。
トカマクは大円周方向のトロイダル磁場と小円周方向のポロイダル磁場からなるトーラス状の磁場配位であり、螺旋状の磁力線が荷電粒子のドリフトを巧妙に打ち消すことで閉じ込めを可能としている一方で、大型のコイルを必要とすることから設計に際しては高いコストを避けることができない。そこで近年急速に研究が進んでいるのがアスペクト比(=プラズマ大半径/プラズマ小半径)の小さい球状トカマクSTである。STは以下で定義される炉心プラズマの経済指標$\beta$を高めることができ、実用炉への採用が有力視されている。
\begin{equation}
	\beta=\frac{P}{B^2/2\mu_0}=\frac{プラズマ熱圧力}{磁気圧力}
\end{equation}
\begin{figure}[H]\centering\includegraphics[scale = 0.3]{fig/ST.png}
	\caption{従来式トカマク(左)とST(右)}\label{fig:ST1}
\end{figure}

\section{磁気リコネクション}
完全電離状態のプラズマ中において反平行の磁力線が互いに接近するとき、電磁誘導の法則から理想的条件においてはその境界に無限大の伝導率をもつ電流シートが形成され、つなぎ替わりは阻止される。しかし実際のプラズマは有限の抵抗率を持つため磁力線はつなぎ替わり、ちょうど伸び切ったゴムひもが勢いよく縮むように、磁場エネルギーは粒子の莫大な運動エネルギーへと変換される\cite{yamada2010}\cite{onobeta}。このとき磁力線が出入りする領域とつなぎ替わる点をそれぞれアウトフロー領域、インフロー領域、X点と呼ぶ。磁気リコネクションと呼ばれるこのエネルギー変換過程は太陽フレアや地球磁気圏などで観測される普遍的物理現象でありながらその機構については未知の部分も多く、X点近傍の局所的な加熱に留まらずアウトフロー領域を通じたプラズマ全体のグローバルな加熱に影響することから、ミクロ・マクロ両視点における統一的な理解が求められている。
\begin{figure}[H]\centering\includegraphics[scale = 0.6]{fig/MR2.png}
	\caption{磁気リコネクション概略図}\label{fig:MR1}
\end{figure}

\subsection{Sweet-Parkerの磁気リコネクションモデル}
つなぎ替わり領域として二次元電流シートを考えるこのモデルは,磁気リコネクションを簡潔かつ定量的に記述する。以下の条件を仮定して,インフロー速度$V_{in}$とアウトフロー速度$V_{out}$を記述できる。\\
(1)磁力線は幅$2\delta$,長さ$2L$($\delta\ll L$)の二次元拡散領域でつなぎ替わる。\\
(2)つなぎ替わり過程は準定常過程である。\\
(3)プラズマは拡散領域で非圧縮性である。\\
(4)プラズマの抵抗率$\eta$は拡散領域外側で$0$, 内側で一定である。\\
(5)インフロー領域の磁場$B_{in}$に対してアウトフロー磁場$B_{out}$が十分大きい。\\
\begin{figure}[H]
	\centering\includegraphics[scale = 0.4]{fig/SW.png}
	\caption{Sweet-Parkerモデルの概略図}\label{fig:SW}
\end{figure}
まずはインフロー領域とアウトフロー領域が非圧縮性によって連続であるから
\begin{equation}
	V_{in}L = V_{out}\delta
\end{equation}
であって,オームの法則は
\begin{equation}
	E + V \times B = \gamma j
\end{equation}
であるからトロイダル電場$E_t$はインフロー領域で以下の通りに表現できる。
\begin{equation}
	E_t = V_{in}B_{in}
\end{equation}
また,アンペールの法則を拡散領域全体に渡って積分すると
\begin{align}
	\nabla \times B &= \mu_0 j&
	\int_C B \cdot ds &= 4L\delta \mu_0 j_t&
\end{align}
となって,$B_{in}<<B_{out}$の仮定より左辺の積分結果が$4LB$と直接表現できるから以下の式が成り立つ。
\begin{equation}
	j_t = \frac{B_{in}}{\mu_0 \delta}
\end{equation}
さらに,電流シートに対して平行な成分と垂直な成分との圧力平衡の式(Bernoulliの式)より
\begin{equation}
\frac{B_0^2}{8\pi}+p_{in}=p_c=\frac{\rho V_{out}}{2}+p_{out}
\label{eq:12}
\end{equation}
と書け,式\ref{eq:12}から$p_{in}=p_{out}$,$V_{out}$はAlfven速度と見なして$V_{out}=V_A=B/\sqrt{4\pi\rho}(\rho=n_iM_i)$となる。\\ 
この議論から,Sweet-Parkerモデルで与えられる特徴的なリコネクション時間は
\begin{equation}
\tau_{rec}=\frac{L}{V_R}=\sqrt{\tau_A\tau_R}=\frac{\tau_R}{\sqrt{S}}\ll\tau_R
\end{equation}
で与えられるが,この時間は$S\gg1$のとき抵抗拡散時間と比較して短い。このモデルを採用したのでは,ソーラーフレアにおけるリコネクションのタイムスケールが数か月と計算され,実際に観測されているタイムスケールより極めて遅いことが実証されている。この乖離に答えるべく近年は磁気リコネクション高速化のメカニズムがPICシミュレーションや実験研究において提唱されており,Hall効果によるイオン拡散領域の発生と運動論効果による異常抵抗がリコネクションを早める要因だと考えられている[\citen{yamada2010}]。


実際の磁気リコネクションは反平行磁力線に加えて電流シート垂直方向の磁場を加えた三次元的現象であり、トカマクやSTでの磁気リコネクションもまたトロイダル磁場がガイド磁場として振る舞いエネルギー変換過程が複雑化する。
高ガイド磁場下でのリコネクションの特徴として、拡散領域全体においてガイド磁場が支配的となり、つなぎ替わりを阻止する方向に誘起されたリコネクション電場によって加速された電子がX点から高速で射出されることと、その電子によりセパラトリクス上に四重極のポテンシャル構造を生じることがある\cite{3drecon}。その際の電子加熱は先に述べたリコネクション電界によるダウンストリーム加熱の他にも、X点近傍のリコネクション電界におけるベータトロン的加熱、電流シート中電子のオーム加熱、セパラトリクス加速\cite{separatrics}、波乗り加速\cite{surf}など様々である。PICシミュレーションにおいても、高速電子の分布が予測されている。

\begin{figure}[H]
	 \centering
	 \includegraphics[scale=0.6]{fig/potential.png}
	 \caption{四重極状ポテンシャルの形成}
	 \label{fig:reconection}
\end{figure}

\section{本研究の目的}
以上に述べた高速電子分布を計測するためには、電子が電場で減速する際に制動放射で放出する軟X線を観測し、これを逆問題として計算機処理すればよい。東大UTSTの実験では図\ref{fig:utst}に示すとおりリコネクション期間初期にアウトフロー領域とセパラトリクス領域において軟X線が観測されており\cite{utst}、またTS-6においても複数のマイクロチャネルプレートMCPを用いた軟X線観測システムとTikhonov-Phillips正則化および最小GCV基準\cite{iwama}を用いた再構成手法によりダウンストリーム加熱が実証されている。しかしこのTS-6実験においてはセパラトリクス加熱、X点加熱などのより局所的な加熱の検出やダウンストリーム加熱における詳細な時間発展計測による加熱機構の説明はなされておらず,発光画像に対するノイズ除去を目的としたフィルタリングや適切な再構成手法の選択といったソフト面の検討も十分ではない。本研究は磁気リコネクションにおける電子加熱の詳細構造をX線画像計測によって再構成することに焦点をあて,
\begin{itemize}
	\item [(1)]加熱領域が最大であるダウンストリーム加熱の時間発展の計測
\end{itemize}
および小さい領域における未知の加熱効果
\begin{itemize}
	\item [(2)]X点加熱の検出
	\item [(3)]セパラトリクス加熱の検出
\end{itemize}
を検出することをテーマとする。第一に最小Fisher情報量による正則化法とNLMフィルタリングを採用した再構成ソフトウェアの開発および性能の検討を行い,第二にTS実験における高エネルギー電子の局所分布と加熱機構の解明を行う。
\begin{figure}[H]
	 \centering
	 \includegraphics[scale=0.6]{fig/utst_separatrix.png}
	 \caption{UTSTで観測されたセパラトリクス加熱\cite{utst}}
	 \label{fig:utst}
\end{figure}